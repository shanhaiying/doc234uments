\documentclass[12pt,letter]{ustcexam}
%\input{greektex}
\newcommand\uppi{\text{\gr p}}

\def\ds{\displaystyle}

\renewcommand{\biaoti}[1]{\begin{spacing}{1.3}\begin{center}\Large\zihao{2}
                          #1\end{center}\end{spacing}\medskip}

\begin{document}
\begin{CJK*}{UTF8}{song}
\CJKindent \CJKtilde % \CJKcaption{GB}
% 以下是将试卷的密封线。
\mifengxian

\pagestyleB
%\juemi
% 中国科学技术大学 试卷标准
%\biaoti{\huge 中国科学技术大学\\[0.2cm]
%     \heiti 2007 -- 2008 学年 第\ 1\ 学期考试试卷B }
\YearB{2007}
\YearE{2008}
\Xueqi{1}
\Kemu{数\ 值\ 计\ 算\ 方\ 法\ }
\MaketitleB   %使用B卷
\Makedefen

%\begin{minipage}{0.9 \textwidth}
%{\zihao{5}
%考试科目:\underline{ 计\ 算\ 方\ 法\ }\hspace{4.93cm}
%得分:\underline{\hspace{2.5cm}}%\hspace{5cm}
%\\
%学生所在系:\underline{\hspace{2.5cm}}\hspace{0.25cm}
%姓名:\underline{\hspace{2.5cm}}\hspace{0.25cm}
%学号:\underline{\hspace{2.5cm}}%\hspace{1cm}
%}
%\end{minipage}
\vspace{1em}
\begin{notice}
\item 答卷前,考生务必将所在系、姓名、学号等填写清楚。
\item 请考生在答卷纸左侧留出装订区域。
\item 本试卷为闭卷考试。共~\numquestions{}~道试题,
      满分~100~分,考试时间~120~分钟。
\item 计算中保留4位小数。
\end{notice}

\begin{problems}
\pingfen{一、填空题}
\qu (6分) 设 $X=(x_1,x_2,x_3)^T$,则如下的公式能否成为向量范数, \\
$|x_1|+3|x_2+x_1|+2|x_3|$\twob\oneb ,
$x_1^2+3|x_2|+2|x_3|$\twob\oneb 。
\begin{sol}
 否\rfen{2},否\rfen{2}
\end{sol}

\qu (3分) 幂法可以求出一个矩阵的哪个特征值?\sixb 。
\begin{sol}
按模最大的 \rfen{3}
\end{sol}

\qu (3分)差商$f[x_0,x_1,x_2,x_3]=(f[x_0,x_1,x_3]-f[x_0,x_2,x_3])/(\sixb)$ 。~
\begin{sol}
 $x_1-x_2$\rfen{3}
\end{sol}

\qu (6分) 设 $ \ds{A=\begin {pmatrix} a & -a \\ -b & b \end {pmatrix}}$,
\ 则 $ \|A\|_1=$ \sixb ,$\|A\|_{\infty}=$ \sixb 。
\begin{sol}
 $|a|+|b|$ \rfen{3}, $2\max{|a|,|b|}$ \rfen{3}
\end{sol}

\qu (6分) 设$l_0(x)$,$l_1(x)$,$l_2(x)$,$l_3(x)$是以互异的$x_0,x_1,x_2,x_3$为
节点的Lagrange插值基函数,\\
则$\displaystyle \sum_{j=0}^{3}l_j(x)(x_j^2+1)=$\sixb。
\begin{sol}
$ x^2+1 $ \rfen3
\end{sol}

\qu (6分) 写出以 $ (\alpha,f(\alpha),f'(\alpha)),(0,f(0)),
(\beta,f(\beta),f'(\beta)) $ 为插值点构造的
插值多项式的截断误差: \\ \\ \tenb\tenb\tenb 。
\begin{sol}
$\dfrac{f^{(5)}(\xi)}{5!}(x-\alpha)^2(x)^2(x-\beta)^2 , \xi\in[a,c]$ (4分)
\end{sol}

\newpage
\pingfen{二、解答题}
\qu (8分) 对数据 
\begin{tabular}{c|cccc} $x_i$  & 1 & 1.5 & 2.0 & 3.0
\\\hline $f(x_i)$ & 2.0 & 3.5 & 2.0 & 1.5
\end{tabular} ,试对它作出$y(x)=a+b x^2$形式的拟合函数。

\vspace*{0.4\textheight}

\qu (8分) 取初值$1.2$,用Newton迭代法求$\sqrt[3]{3}$的近似值(最多只计算5步)。

%\vspace*{0.4\textheight}

\newpage
\qu (9分)计算矩阵
$A=\begin{pmatrix} 5 & 2 & 4 \\ 10 & 5 & 8 \\ -5 & -2 & -1 \end{pmatrix}$的
Dolittle分解,即分解为一个单位下三角阵和一个上三角阵的积
\begin{sol}
$
L=
\left(\begin{array}{ccc}
1 &  &  \\
2 & 1 &  \\
-1 & 0 & 1
\end{array}\right)
U=
\left(\begin{array}{ccc}
5 & 2 & 4 \\
 & 1 & 0 \\
 &  & 3
\end{array}\right) 
$
\end{sol}

\vspace*{0.4\textheight}

\qu (15分) 设有数据\begin{tabular}{c|cccc} $x_i$  & $ -1$ &0.5 & 1.0 &2.0
\\\hline $f(x_i)$ & $0$ & 3.0 &2.0 & 1.5
\end{tabular}
\begin{enumerate}
  \item 做出差商表
  \item 写出Newton型的插值多项式
\end{enumerate}

\newpage
\qu (18分) $A=\begin{pmatrix}1 & c \\ 4c & 1 \end{pmatrix}$ 分别写出解线性方程组
$Ax=b$的
\begin{enumerate}
  \item Jacobi和Gauss-Seidel的迭代格式和迭代矩阵;
  \item 对两种格式讨论它们收敛的充要条件。
\end{enumerate}

%\vspace*{0.4\textheight}
\newpage
\qu (12分)
对常微分方程的初值问题
$\left\{\begin{array}{l} \dfrac{dy}{dx}=f(x,y) \\ y(a)=y_0\end{array}\right. 
 (a\leqslant x \leqslant b)$的等价积分形式
$$y(x_{n+1})=y(x_{n-p})+\int_{x_{n-p}}^{x_{n+1}}y'(x)dx$$
建立等距节点下的$p=1,q=2$的显格式。

\newpage

\end{problems}


\clearpage
\daan{答案}
\clearpage

\end{CJK*}

\end{document}
