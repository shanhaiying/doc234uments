\documentclass[12pt,letter]{ustcexam}
%\input{greektex}
\newcommand\uppi{\text{\gr p}}

\def\ds{\displaystyle}

%\renewcommand{\biaoti}[1]{\begin{spacing}{1.3}\begin{center}\Large\zihao{2}
%                          #1\end{center}\end{spacing}\medskip}

\begin{document}
%\begin{CJK*}{UTF8}{song}
%\CJKindent \CJKtilde % \CJKcaption{GB}
% 以下是将试卷的密封线。
\mifengxian

\pagestyleB
%\juemi
% 中国科学技术大学 试卷标准
%\biaoti{\huge 中国科学技术大学\\[0.2cm]
%     \heiti 2007 -- 2008 学年 第\ 1\ 学期考试试卷B }
\YearB{2008}
\YearE{2009}
\Xueqi{1}
\Kemu{数\ 值\ 计\ 算\ 方\ 法\ }
%\MaketitleB   %使用B卷
\Makedefen

%\begin{minipage}{0.9 \textwidth}
%{\zihao{5}
%考试科目:\underline{ 计\ 算\ 方\ 法\ }\hspace{4.93cm}
%得分:\underline{\hspace{2.5cm}}%\hspace{5cm}
%\\
%学生所在系:\underline{\hspace{2.5cm}}\hspace{0.25cm}
%姓名:\underline{\hspace{2.5cm}}\hspace{0.25cm}
%学号:\underline{\hspace{2.5cm}}%\hspace{1cm}
%}
%\end{minipage}
\vspace{1em}
\begin{notice}
\item 答卷前,考生务必将所在系、姓名、学号等填写清楚。
\item 请考生在答卷纸左侧留出装订区域。
\item 本试卷为闭卷考试。共~\numquestions{}~道试题,
      满分~100~分,考试时间~120~分钟。
\item 计算结果保留4位小数。
\end{notice}

\begin{problems}
\pingfen{一、填空题}
\qu (6分) 设 $X=(x_1,x_2,x_3)^T$,则如下的公式能否成为向量范数, \\
$|x_1|+2|x_2|+|x_1+3x_3|$\twob\oneb ,
$|x_1|+|x_2+2x_3|$\twob\oneb 。
\begin{sol}
 否\rfen{2},否\rfen{2}
\end{sol}

\qu (6分) 设 $ f(x)=2x^5+7 $,则$f[-1,0,1]=$\twob\twob ,
$f[1,2,3,4,5,6]=$ \twob\twob
\begin{sol}
 2\rfen{2},5\rfen{2}
\end{sol}

\question (6分) 设 $ \ds{A=\begin {pmatrix} 8 & 3 & 1 \\
2 & -2 & -7 \\ 0 & 5 & 3 \end {pmatrix}}$,\  则 $ \|A\|_1=$ \sixb ,
$ \|A\|_\infty =$ \sixb 。
\begin{sol}
10\rfen{2},12\rfen{2}
\end{sol}

\qu (6分) 用Gauss消元法解线性方程组的可行性条件是矩阵$A$\tenb\fourb,
用什么方法可以有效地克服这个缺陷?\fourb\fourb。
\begin{sol}
3  \rfen3
\end{sol}

\question (6分) 用Jacobi方法求矩阵 $ \ds{A=\begin {pmatrix} 5 & 1 & 4 \\
1 & 0 & 2 \\ 4 & 2 & 3 \end {pmatrix}}$的所有特征值,设第一步的Givens旋转阵是
$Q(p,q,\theta)$,则 $p=$\fourb ,$q=$\fourb 。
\begin{sol}
10\rfen{2},12\rfen{2}
\end{sol}

\qu (6分) 三次样条函数是指在每个小区间上是次数不高于\twob 次多项式,
但整体具有\twob 阶连续导数的函数。
\begin{sol}
3,2
\end{sol}

\qu (6分) 写出求$\sqrt[3]{5}$的Nowton迭代格式。\tenb\tenb
\begin{sol}
$x_{n+1}=$
\end{sol}

\qu (6分) 写出以$(a,f(a),f'(a)),(0,f(0),f'(0)),(b,f(b))$为插值条件构造的插值多项式的误差\\ \sixb\sixb\tenb 。
\begin{sol}
a
\end{sol}

\newpage
\pingfen{二、解答题}
\qu (8分) 对数据 
\begin{tabular}{c|cccc} $x_i$  & 1 & 2 & 3 & 4
\\\hline $f(x_i)$ & 1.1 & 2.5 & 2.2 & 1.4
\end{tabular} ,试对它作出$y(x)=a+b x^2$形式的拟合函数。

\vspace*{0.4\textheight}

\qu (5分) 求数值积分公式
$
 \displaystyle \int_a^b f(x) dx=(b-a)f\left(\dfrac{a+b}{2}\right)
$
的误差。

%\vspace*{0.4\textheight}

\newpage
\qu (12分)  写出用Gauss-Seidel方法求解下列方程组
$$\begin {cases} 6x_1+x_2-4x_3=5\\ 2 x_1+5x_2+ x_3=-4 \\ x_1-2x_2+3x_3=2
\end{cases}$$
1) 迭代格式(3分);  2) 迭代矩阵(6分); 3) 讨论迭代矩阵是否收敛(3分)?
\begin{sol}
1)(5分)
$
\left\{\begin{array}{l}
x_1^{k+1}=(-x_2^k+x_3^k+5)/10 \\
x_2^{k+1}=(-x_1^{k+1}-x_3^k-4)/5 \\
x_3^{k+1}=(-x_1^{k+1}-x_2^{k+1}+2)/2
\end{array}\right.
$ \\
2)(5分)
$
\left(\begin{array}{ccc}
0 & -1/10 & 1/10 \\
0 & 1/50 & -11/50 \\
0 & 1/25 & 3/50
\end{array}\right)
$ \\
3) (5分)谱半径为$(2\pm i\sqrt{21})/50,0$或$\|S\|_1=19/50<1$
   或$\|S\|_{\infty}=12/50<1$
\end{sol}

%\vspace*{0.4\textheight}
\newpage
\qu (12分)用$LU$分解,其中$L$为单位下三角阵,$U$为上三角阵,求解下列方程组
\[  \qquad \begin{cases}
2x_1+ 0.5x_2 + x_3&=0.5 \\
2x_1+ 1.5x_2 + x_3&=-0.5\\
x_1-0.75x_2 + 3.5x_3&=4.25
\end{cases}\]
\begin{sol}
$
L=
\left(\begin{array}{ccc}
1 &  &  \\
1 & 1 &  \\
0.5 & -1 & 1
\end{array}\right)
U=
\left(\begin{array}{ccc}
2 & 0.5  & 1 \\
 & 1 & 0  \\
 &  & 3
\end{array}\right)
$
(9分)\\
$
x=
\left(\begin{array}{l}
0 \\
-1 \\
1
\end{array}\right)
$(3分)
\end{sol}

\newpage

\qu (15分)
\begin{enumerate}
\item (6分) 考虑常微分方程初值问题
$$ \left\{\begin{array}{l}y'=f(x,y)=y^2 x \mbox{  ,  } 0\leqslant x\leqslant 1 \\
y(0)=0
\end{array}\right.$$
用如下的改进Euler格式求$y(0.05)$的近似,取步长$h=0.05$
$$
\left\{\begin{array}{l}
y_{n+1}=y_n+\dfrac{h}{2}(k_1+k_2) \\
k_1=f(x_n,y_n) \\
k_2=f(x_n+h,y_n+hK_1) 
\end{array}\right.
$$
\item (5分) 求解微分方程$y'(x)=f(x,y)$的线性多步法
$$
  y_{n+1}=-\dfrac{3}{2}y_n+3y_{n-1}-\dfrac{1}{2}y_{n-1}+3hf_n
$$
的局部截断误差,其中$f_n=f(x_n,y_n)$
\item (4分) 若用上面的多步法来求解的话,起步计算可以用什么格式?请简述理由。
\end{enumerate}
\begin{sol}
1)
\begin{tabular}{|l|l|l}
\hline
k1 & 1 & \rfen2 \\
k2 & 1.0250 & \rfen2 \\
y(0.05) & 0.0512 & \rfen2 \\
\hline
\end{tabular}

2) $O(h^4)$

3) 2阶Runge-Kutta格式,因为它的局部截断误差为$O(h^3)$,不会影响整体的$O(h^3)$的精度
\end{sol}

\vspace*{0.25\textheight}

\newpage

\end{problems}


\clearpage
\daan{答案}
\clearpage

%\end{CJK*}

\end{document}
