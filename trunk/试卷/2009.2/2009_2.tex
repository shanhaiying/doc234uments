\documentclass[12pt,letter]{ustcexam}
%\input{greektex}
\newcommand\uppi{\text{\gr p}}

\def\ds{\displaystyle}

%\renewcommand{\biaoti}[1]{\begin{spacing}{1.3}\begin{center}\Large\zihao{2}
%                          #1\end{center}\end{spacing}\medskip}

\begin{document}
%\begin{CJK*}{UTF8}{song}
%  \CJKindent \CJKtilde % \CJKcaption{GB}
% 以下是将试卷的密封线。
\mifengxian

\pagestyleB
%\juemi
% 中国科学技术大学 试卷标准
%\biaoti{\huge 中国科学技术大学\\[0.2cm]
%     \heiti 2007 -- 2008 学年 第\ 1\ 学期考试试卷B }
\YearB{2008}
\YearE{2009}
\Xueqi{1}
\Kemu{数\ 值\ 计\ 算\ 方\ 法\ }
\MaketitleB   %使用B卷
\Makedefen

%\begin{minipage}{0.9 \textwidth}
%{\zihao{5}
%考试科目:\underline{ 计\ 算\ 方\ 法\ }\hspace{4.93cm}
%得分:\underline{\hspace{2.5cm}}%\hspace{5cm}
%\\
%学生所在系:\underline{\hspace{2.5cm}}\hspace{0.25cm}
%姓名:\underline{\hspace{2.5cm}}\hspace{0.25cm}
%学号:\underline{\hspace{2.5cm}}%\hspace{1cm}
%}
%\end{minipage}
\vspace{1em}
\begin{notice}
\item 答卷前,考生务必将所在系、姓名、学号等填写清楚。
\item 请考生在答卷纸左侧留出装订区域。
\item 本试卷为闭卷考试。共~\numquestions{}~道试题,
      满分~100~分,考试时间~120~分钟。
\item 计算结果保留4位小数。
\end{notice}

\begin{problems}
\pingfen{一、填空题}
\qu (6分) 设 $X=(x_1,x_2,x_3)^T$,则如下的公式能否成为向量范数, \\
$|x_1|+2|x_2|+|x_1-3x_3|$\twob\oneb ,
$|x_1|+|x_2+2x_3|$\twob\oneb 。
\begin{sol}
 否\rfen{3},否\rfen{3}
\end{sol}

\question (6分) 设 $ \ds{A=\begin {pmatrix} -8 & 3 & 2 \\
2 & -2 & -9 \\ -1 & 5 & 3 \end {pmatrix}}$,\  则 $ \|A\|_1=$ \sixb ,
$ \|A\|_\infty =$ \sixb 。
\begin{sol}
14\rfen{2},13\rfen{2}
\end{sol}

\qu (3分) 用直接分解法解线性方程组的可行性条件是矩阵$A$\tenb\fourb。
\begin{sol}
3  \rfen3
\end{sol}

\question (3分) QR方法可以用来求什么类型矩阵的所有特征值?\fourb\fourb 。
\begin{sol}
10\rfen{2},12\rfen{2}
\end{sol}

\qu (6分) $\{l_i(x),i=0,\cdots 7\}$是节点$\{0,1,\cdots,7\}$上的Lagrange插值基函数, \\
则$\displaystyle \sum_{i=0}^7l_i(x)(2 i^5+7)=$\twob\fourb。
\begin{sol}
3,2
\end{sol}

\qu (6分) 数值求积公式$ \int_a^b f(x) dx=(b-a)f\left(b\right) $的误差为\tenb。

\qu (6分) 写出以$(a,f(a)),(0,f(0)),(b,f(b),f'(b))$为插值条件构造的插值多项式的误差\\ \sixb\sixb\tenb 。
\begin{sol}
a
\end{sol}

\question (9分) 设 $ \ds{A=\begin {pmatrix} 2 & -1 & 0 \\
3 & -2 & -3 \\ 0 & 4 & -10 \end {pmatrix}}$,\  则 
$ A=\begin{pmatrix} \twob & 0 & 0 \\ \twob & \twob & 0 \\ \twob & \twob &\twob
\end{pmatrix} \begin{pmatrix} 1 & \twob & \twob \\ 0 & 1 & \twob \\ 0 & 0 & 1 
\end{pmatrix} $ 。
\begin{sol}
$ A=\begin{pmatrix} 2 & 0 & 0 \\ 3 & 1 & 0 \\ 0 & 4 & 2\end{pmatrix}
 \begin{pmatrix} 1 & -1 & 0 \\ 0 & 1 & -3 \\ 0 & 0 & 1 
\end{pmatrix} $ 。
\end{sol}

\newpage
\pingfen{二、解答题}
\qu (12分) 对数据 
\begin{tabular}{c|cccc} $x_i$  & 1 & 2 & 3 & 4
\\\hline $f(x_i)$ & 1.1 & 2.5 & 2.2 & 1.4
\end{tabular} ,
构造差商表,写出相应的Newton插值多项式,并求出$f(2.5)$的近似值。

\vspace*{0.4\textheight}

\qu (12分) 取初值$2.0$,用Newton迭代法求$\sqrt[3]{5}$的近似(最多只计算3步)。
%\vspace*{0.4\textheight}

\newpage
\qu (12分)  写出用Gauss-Seidel方法求解下列方程组
$$\begin {cases} 8x_1+2x_2-4x_3=5\\ 2 x_1+5x_2+ x_3=-4 \\ -x_1-3x_2+4x_3=2
\end{cases}$$
1) 迭代格式(3分);  2) 迭代矩阵(6分); 3) 讨论迭代矩阵是否收敛(3分)?
\begin{sol}
1)(5分)
$
\left\{\begin{array}{l}
x_1^{k+1}=(-x_2^k+x_3^k+5)/10 \\
x_2^{k+1}=(-x_1^{k+1}-x_3^k-4)/5 \\
x_3^{k+1}=(-x_1^{k+1}-x_2^{k+1}+2)/2
\end{array}\right.
$ \\
2)(5分)
$
\left(\begin{array}{ccc}
0 & -1/10 & 1/10 \\
0 & 1/50 & -11/50 \\
0 & 1/25 & 3/50
\end{array}\right)
$ \\
3) (5分)谱半径为$(2\pm i\sqrt{21})/50,0$或$\|S\|_1=19/50<1$
   或$\|S\|_{\infty}=12/50<1$
\end{sol}

%\vspace*{0.4\textheight}
\newpage

\qu (5分) 考虑常微分方程初值问题
$$ \left\{\begin{array}{l}y'=x^2+y \mbox{  ,  } 0\leqslant x\leqslant 1 \\
y(0)=1
\end{array}\right.$$
用如下4阶Runge-Kutta格式求$y(0.05)$的近似,取步长$h=0.05$
$$
\left\{\begin{array}{l}
y_{n+1}=y_n+\dfrac{h}{6}(K_1+2K_2+2K_3+K_4) \\
K_1=f(x_n,y_n) \\
k_2=f(x_n+\dfrac{1}{2}h,y_n+\dfrac{1}{2}hK_1) \\
k_2=f(x_n+\dfrac{1}{2}h,y_n+\dfrac{1}{2}hK_2) \\
k_2=f(x_n+h,y_n+hK_3)
\end{array}\right.
$$


\vspace*{0.25\textheight}

\newpage

\qu (14分)求系数$A,B,C,\alpha$,使得数值积分公式
$$
  \int_{-1}^2f(x)dx=Af(\alpha)+Bf(0)+Cf(\alpha)
$$
具有尽可能高的代数精度,并求出该公式的代数精度。
\begin{sol}
haha,我也不知道答案
\end{sol}

\newpage

\end{problems}


\clearpage
\daan{答案}
\clearpage

%\end{CJK*}

\end{document}
