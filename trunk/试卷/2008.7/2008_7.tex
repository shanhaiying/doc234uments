\documentclass[12pt,letter]{ustcexam}
%\input{greektex}
\newcommand\uppi{\text{\gr p}}

\def\ds{\displaystyle}

\renewcommand{\biaoti}[1]{\begin{spacing}{1.3}\begin{center}\Large\zihao{2}
                          #1\end{center}\end{spacing}\medskip}

\begin{document}
\begin{CJK*}{UTF8}{song}
\CJKindent \CJKtilde % \CJKcaption{GB}
% 以下是将试卷的密封线。
\mifengxian

\pagestyleB
%\juemi
% 中国科学技术大学 试卷标准
%\biaoti{\huge 中国科学技术大学\\[0.2cm]
%     \heiti 2007 -- 2008 学年 第\ 1\ 学期考试试卷B }
\YearB{2007}
\YearE{2008}
\Xueqi{2}
\Kemu{数\ 值\ 计\ 算\ 方\ 法\ }
%\MaketitleB   %使用B卷
\Makedefen

%\begin{minipage}{0.9 \textwidth}
%{\zihao{5}
%考试科目:\underline{ 计\ 算\ 方\ 法\ }\hspace{4.93cm}
%得分:\underline{\hspace{2.5cm}}%\hspace{5cm}
%\\
%学生所在系:\underline{\hspace{2.5cm}}\hspace{0.25cm}
%姓名:\underline{\hspace{2.5cm}}\hspace{0.25cm}
%学号:\underline{\hspace{2.5cm}}%\hspace{1cm}
%}
%\end{minipage}
\vspace{1em}
\begin{notice}
\item 答卷前,考生务必将所在系、姓名、学号等填写清楚。
\item 请考生在答卷纸左侧留出装订区域。
\item 本试卷为闭卷考试。共~\numquestions{}~道试题,
      满分~100~分,考试时间~120~分钟。
\item 计算中保留4位小数。
\end{notice}

\begin{problems}
\pingfen{一、填空题}
\qu (6分)设$f(x)=5x^3+3x+4$,则$f[-1,0,1]=\twob$ ,$f[0,1,2,3]=$ \sixb 。~
\begin{sol}
6\rfen{3},3\rfen{3}
\end{sol}

\qu (6分) 设 $ \ds{A=\begin {pmatrix} 1 & 2 \\ -2 & 3 \end {pmatrix}}$,
\ 则 $ \rho(A)=$ \sixb , $ Cond_1(A) =$ \sixb 。
\begin{sol}
$\sqrt 7$\rfen{3},25/7\rfen{3}
\end{sol}

\qu (3分)对节点组$\{x_0,x_1,x_2,x_3\}$,已知$f(x_3),f[x_2,x_3],f[x_1,x_2,x_3],f[x_0,x_1,x_2,x_3]$,给出相应的Newton插值多项式\sixb\sixb 。~
\begin{sol}
 $f(x_3)+(x-x_3)f[x_2,x_3] $
\end{sol}


\qu (6分) 写出以 $ (\alpha,f(\alpha),f'(\alpha)),(0,f(0)),
(\beta,f(\beta),f'(\beta),f''(\beta)) $ 为插值点构造的
插值多项式的截断误差: \\ \\ \tenb\tenb\tenb 。
\begin{sol}
$\dfrac{f^{(6)}(\xi)}{6!}(x-\alpha)^2(x)^2(x-\beta)^3 , \xi\in[a,c]$ (4分)
\end{sol}

\qu (6分) 给出Gauss-Jordan消元法解$n$阶线性方程组的乘除法的计算量\sixb\sixb
\begin{sol}
$n^3$
\end{sol}

\newpage
\pingfen{二、解答题}
%\qu (7分) 矩阵$A=\begin{pmatrix} 9.4705 & 4.70588 & 1.94118 \\
%      2.11765 & -8.82353 & -7.76471 \\
%      -2.82353 & -1.56863 & 0.35294 \end{pmatrix}$
%在用规范的幂法求矩阵$A$的按模最大特征值时,收敛后有如下计算结果,
%请计算矩阵的按模最大特征值及相应的特征向量。
%\begin{table}[!hp]
%\begin{center}
%\begin{tabular}{|c|}
%  \hline
%  $\cdots$ \\ \hline
%  (0.737859, -0.628547, -0.245953) \\ \hline
%  (0.363819, 0.923541, -0.121273) \\ \hline
%  (0.737859, -0.628547, -0.245953) \\ \hline
%  (0.363819, 0.923541, -0.121273) \\
%  \hline
%\end{tabular}
%\end{center}
%\end{table}

%\vspace*{0.4\textheight}

%\newpage

\qu (8分) 对数据 
\begin{tabular}{c|cccc} $x_i$  & 1 & 1.5 & 2.0 & 3.0
\\\hline $f(x_i)$ & 2.0 & 3.5 & 2.0 & 1.5
\end{tabular} ,试对它作出$y(x)=a +b \sqrt{x}$形式的拟合函数。

\vspace*{0.4\textheight}

\qu (8分) 取初值$1.2$,用Newton迭代法求$\sqrt[3]{5}$的近似值(最多只计算5步)。

%\vspace*{0.4\textheight}

\newpage

\qu (8分) 设有数据
\begin{tabular}{c|ccccccc} 
\hline
 $x_i$    & 0.5 & 1.0 & 1.5 & 2.5 & 3.5 & 4.0 & 4.5 \\
\hline 
 $f(x_i)$ &   0 & 3.0 & 2.0 & 1.5 & 2.0 & 2.4 & 3.0 \\
\hline
\end{tabular}
用复化Simpson方法求数值积分
\begin{sol}
h
\end{sol}

\vspace*{0.4\textheight}

\qu (9分)计算矩阵
$A=\begin{pmatrix} 2 & 1 &-3 \\ 3 & 4 & -0.5 \\ 4 & -4.25 &-14 \end{pmatrix}$的
Dolittle分解,即分解为一个单位下三角阵和一个上三角阵的积
\begin{sol}
$
L=
\left(\begin{array}{ccc}
1 &  &  \\
1.5 & 1 &  \\
2 & -2.5 & 1
\end{array}\right)
U=
\left(\begin{array}{ccc}
 2 & 1 & -3 \\
   & 2.5 & 4 \\
   &  & 2
\end{array}\right) 
$
\end{sol}

\vspace*{0.4\textheight}

\newpage
\qu (18分) $A=\begin{pmatrix} 1 & c \\ 2c & 3 \end{pmatrix}$ 
分别写出解线性方程组$Ax=b$的
\begin{enumerate}
  \item Jacobi和Gauss-Seidel的迭代格式和迭代矩阵;
  \item 对两种格式讨论它们收敛的充要条件。
\end{enumerate}

%\vspace*{0.4\textheight}
\newpage
\qu (22分)
对常微分方程的初值问题
$\left\{\begin{array}{l} \dfrac{dy}{dx}=f(x,y) \\ y(a)=y_0\end{array}\right. 
 (a\leqslant x \leqslant b)$的等价积分形式
$$y(x_{n+1})=y(x_{n-p})+\int_{x_{n-p}}^{x_{n+1}}y'(x)dx$$
\begin{enumerate}
  \item 建立等距节点下的$p=1,q=2$的显格式。
  \item 求该格式的局部截断误差
  \item 若$y_1^{(h)}$是以初值$y_0$步长$h$用该格式计算的$y(a+h)$的近似值,
  $y_1^{(h/2)}$是初值$y_0$步长$h/2$计算的$y(a+h)$的近似,请给出$y_1^{(h/2)}$
  的事后误差估计
\end{enumerate}

\newpage

\end{problems}


\clearpage
\daan{答案}
\clearpage

\end{CJK*}

\end{document}
