\documentclass[12pt,letter]{ustcexam}
%\input{greektex}
\newcommand\uppi{\text{\gr p}}

\def\ds{\displaystyle}

%\renewcommand{\biaoti}[1]{\begin{spacing}{1.3}\begin{center}\Large\zihao{2}
%                          #1\end{center}\end{spacing}\medskip}

\begin{document}
%\begin{CJK*}{UTF8}{song}
%\CJKindent \CJKtilde % \CJKcaption{GB}
% 以下是将试卷的密封线。
\mifengxian

\pagestyleB
%\juemi
% 中国科学技术大学 试卷标准
%\biaoti{\huge 中国科学技术大学\\[0.2cm]
%     \heiti 2007 -- 2008 学年 第\ 1\ 学期考试试卷B }
\YearB{2007}
\YearE{2008}
\Xueqi{2}
\Kemu{数\ 值\ 计\ 算\ 方\ 法\ }
\MaketitleB   %使用B卷
\Makedefen

%\begin{minipage}{0.9 \textwidth}
%{\zihao{5}
%考试科目:\underline{ 计\ 算\ 方\ 法\ }\hspace{4.93cm}
%得分:\underline{\hspace{2.5cm}}%\hspace{5cm}
%\\
%学生所在系:\underline{\hspace{2.5cm}}\hspace{0.25cm}
%姓名:\underline{\hspace{2.5cm}}\hspace{0.25cm}
%学号:\underline{\hspace{2.5cm}}%\hspace{1cm}
%}
%\end{minipage}
\vspace{1em}
\begin{notice}
\item 答卷前,考生务必将所在系、姓名、学号等填写清楚。
\item 请考生在答卷纸左侧留出装订区域。
\item 本试卷为闭卷考试。共~\numquestions{}~道试题,
      满分~100~分,考试时间~120~分钟。
\item 计算结果保留4位小数。
\end{notice}

\begin{problems}
\pingfen{一、填空题}
\qu (6分) 设 $X=(x_1,x_2,x_3)^T$,则如下的公式能否成为向量范数, \\
$|x_1|+|x_2+x_1|+|x_3|$\twob\oneb ,
$x_1^2+|x_2|+2|x_3|$\twob\oneb 。
\begin{sol}
 否\rfen{2},否\rfen{2}
\end{sol}

\qu (6分) 设 $ f(x)=2x^3+3x+5 $,则$f[-1,0,1]=$\twob\twob ,$ f[0,1,2,3]=$ \twob\twob
\begin{sol}
 2\rfen{2},5\rfen{2}
\end{sol}

\question (6分) 设 $ \ds{A=\begin {pmatrix} 1 & 3 & 1 \\
6 & -2 & 2 \\ 3 & 1 & 7 \end {pmatrix}}$,\  则 $ \|A\|_1=$ \sixb ,
$ \|A\|_\infty =$ \sixb 。
\begin{sol}
10\rfen{2},12\rfen{2}
\end{sol}

\qu (6分) 用追赶法解$n$阶3对角方程组的乘除法的计算量为$a O(n^b)$,则$a=$\fourb $b=$\fourb。
\begin{sol}
3  \rfen3
\end{sol}

\qu (6分) 写出以$(a,f(a),f'(a)),(0,f(0)),(b,f(b))$为插值条件构造的插值多项式的误差\sixb\sixb\tenb 。
\begin{sol}
a
\end{sol}

\newpage
\pingfen{二、解答题}
\qu (10分) 对数据 
\begin{tabular}{c|cccc} $x_i$  & 1 & 2.25 & 4.0 & 4.41
\\\hline $f(x_i)$ & 1.1 & 2.5 & 2.2 & 1.4
\end{tabular} ,试对它作出$y(x)=a+b \sqrt{x}$形式的拟合函数。

\vspace*{0.4\textheight}

\qu (10分) 取初值$1.2$,用Newton迭代法求$\sqrt[5]{4}$的近似值(计算3步)。

%\vspace*{0.4\textheight}

\newpage

\qu (10分) 考虑常微分方程初值问题
$$ \left\{\begin{array}{l}y'=e^x \cos y \mbox{  ,  } 0\leqslant x\leqslant 1 \\
y(0)=0
\end{array}\right.$$
用如下4阶Runge-Kutta格式求$y(0.05)$的近似,取步长$h=0.05$
$$
\left\{\begin{array}{l}
y_{n+1}=y_n+\dfrac{h}{6}(K_1+2K_2+2K_3+K_4) \\
K_1=f(x_n,y_n) \\
k_2=f(x_n+\dfrac{1}{2}h,y_n+\dfrac{1}{2}hK_1) \\
k_2=f(x_n+\dfrac{1}{2}h,y_n+\dfrac{1}{2}hK_2) \\
k_2=f(x_n+h,y_n+hK_3)
\end{array}\right.
$$
\begin{sol}
\begin{tabular}{|l|l|l}
\hline
k1 & 1 & \rfen2 \\
k2 & 1.0250 & \rfen2 \\
k3 & 1.0250 & \rfen2 \\
k4 & 1.0499 & \rfen2 \\
y(0.05) & 0.0512 & \rfen2 \\
k1 & 1.0499 & \rfen1 \\
k2 & 1.0747 & \rfen1 \\
k3 & 1.0746 & \rfen1 \\
k4 & 1.0991 & \rfen1 \\
y(0.1) & 0.1050 & \rfen1 \\
\hline
\end{tabular}
\end{sol}

\vspace*{0.25\textheight}

\qu (10分) 给定求积公式
$$\int^1_{-1}f(x)dx \thickapprox Af(-\frac{1}{2})+Bf(0)+Cf(\frac{1}{2})$$
试求$A,B,C$使其具有尽可能高的代数精度,并指出所达到的代数精度。
\begin{sol}
$\left\{\begin{array}{l}
A+B+C=2=\int_{-1}^{1}1dx \\
-\dfrac{1}{2}A+\dfrac{1}{2}C=0=\int_{-1}^{1}x dx \\
\dfrac{1}{4}A+\dfrac{1}{4}C=2=\int_{-1}^{1}x^2dx \\
\end{array}\right.$
解得$A=C=\dfrac{4}{3},B=-\dfrac{2}{3}$ \rfen{10}\\
$\int_{-1}^{1}x^3dx=0=-\dfrac{1}{8}A+\dfrac{1}{8}C$ \rfen{2},
$\int_{-1}^{1}x^4dx=\dfrac{2}{5}\neq \dfrac{1}{16}A+\dfrac{1}{16}C
=\dfrac{1}{6}$ \rfen{2},代数精度为$4$ \rfen{1}
\end{sol}

\newpage
\qu (15分)  写出用Gauss-Seidel方法求解下列方程组
$$\begin {cases} 10x_1+x_2-4x_3=5\\ 2 x_1+5x_2+ x_3=-4 \\ x_1+x_2+5x_3=2
\end{cases}$$
1) 迭代格式;  2) 迭代矩阵; 3) 讨论迭代矩阵是否收敛?
\begin{sol}
1)(5分)
$
\left\{\begin{array}{l}
x_1^{k+1}=(-x_2^k+x_3^k+5)/10 \\
x_2^{k+1}=(-x_1^{k+1}-x_3^k-4)/5 \\
x_3^{k+1}=(-x_1^{k+1}-x_2^{k+1}+2)/2
\end{array}\right.
$ \\
2)(5分)
$
\left(\begin{array}{ccc}
0 & -1/10 & 1/10 \\
0 & 1/50 & -11/50 \\
0 & 1/25 & 3/50
\end{array}\right)
$ \\
3) (5分)谱半径为$(2\pm i\sqrt{21})/50,0$或$\|S\|_1=19/50<1$
   或$\|S\|_{\infty}=12/50<1$
\end{sol}

%\vspace*{0.4\textheight}
\newpage
\qu (15分)用 $LDL^T$分解求解下列方程组
\[  \qquad \begin{cases}
2x_1+ 4x_2 - 2x_3&=-6 \\
4x_1+ 12x_2 + 8x_3&=12\\
-2x_1+ 8x_2 + 39x_3&=80
\end{cases}\]
\begin{sol}
$
L=
\left(\begin{array}{ccc}
1 &  &  \\
-1/2 & 1 &  \\
-1/3 & 4/13 & 1
\end{array}\right)
D=
\left(\begin{array}{ccc}
-6 &  &  \\
 & 13/2 &  \\
 &  & 236/39
\end{array}\right) \\ \mbox{或}\\
L=
\left(\begin{array}{ccc}
1 &  &  \\
-0.5 & 1 &  \\
-0.3333 & 0.3077 & 1
\end{array}\right)
D=
\left(\begin{array}{ccc}
-6 &  &  \\
 & 6.5 &  \\
 &  & 6.0513
\end{array}\right)
$
(8分)\\
$
\left\{\begin{array}{l}
Ly=b \\
Dz=y \\
L^Tx=z
\end{array}\right.
=>
y=
\left(\begin{array}{l}
-5 \\
35/2 \\
-236/39
\end{array}\right)
z=
\left(\begin{array}{l}
5/6 \\
35/13 \\
-1
\end{array}\right)
x=
\left(\begin{array}{l}
2 \\
3 \\
-1
\end{array}\right)
\\ \mbox{或} \\
y=
\left(\begin{array}{l}
-5 \\
17.5 \\
-6.0513
\end{array}\right)
z=
\left(\begin{array}{l}
0.8333 \\
2.6923 \\
-1.0
\end{array}\right)
x=
\left(\begin{array}{l}
2.0 \\
3.0 \\
-1.0
\end{array}\right)
$(6分)
\end{sol}

\newpage

\end{problems}


\clearpage
\daan{答案}
\clearpage

%\end{CJK*}

\end{document}
